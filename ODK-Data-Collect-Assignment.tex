\documentclass[options]{article}

 \usepackage[
    top    = 2.75cm,
    bottom = 2.50cm,
    left   = 4.00cm,
    right  = 3.50cm]{geometry}

\usepackage[parfill]{parskip}

\title{Do programmers or Computer science students work at night?}
\author{Lugya Ahmed \thanks{supervisor: Ernest Mwebaze}}
\date{%
    Makerere University\\%
    May 19, 2017
}


\begin{document}
\begin{titlepage}
\maketitle
\end{titlepage}





\section{\textbf{ Introduction}} 
Computer science is a discipline that spans theory and practice. It requires thinking both in abstract terms and in concrete terms. The practical side of computing can be seen everywhere. Nowadays, practically everyone is a computer user, and many people are even computer programmers. Getting computers to do what you want them to do requires intensive hands-on experience. But computer science can be seen on a higher level, as a science of problem solving. Computer scientists must be adept at modeling and analyzing problems. They must also be able to design solutions and verify that they are correct. Problem solving requires precision, creativity, and careful reasoning. 


\subsection{\textbf{Background}}
Computer science is a discipline that involves the understanding and design of computers and computational processes.A professional computer scientist must have a firm foundation in the crucial areas of the field and will most likely have an in-depth knowledge in one or more of the other areas of the discipline, depending upon the person's particular area of practice. Thus, a well educated computer scientist should be able to apply the fundamental concepts and techniques of computation, algorithms, and computer design to a specific design problem.  \bigbreak

The work includes detailing of specifications, analysis of the problem, and provides a design that functions as desired, has satisfactory performance, is reliable and maintainable, and meets desired cost criteria. Clearly, the computer scientist must not only have sufficient training in the computer science areas to be able to accomplish such tasks, but must also have a firm understanding in areas of mathematics and science, as well as a broad education in liberal studies to provide a basis for understanding the societal implications of the work being performed. \bigbreak



\subsection{\textbf{Problem Statement}}
You can divide programmers into roughly two groups - freelancers, founders, indies who set their own schedules, and those whose schedule is dictated by the organization they’re in.And the mission of this research is to determine whether programmers do in fact work at night or is this just some myth perpetuated by people in society.
\subsection{\textbf{Objectives}}


\subsubsection{\textbf{Main Objective}} 
The main goal of this project is to determine whether programmers or Computer scientists work at night or not.


\subsubsection{\textbf{Specific Objectives}}

\begin{itemize}
  \item To collect all the data necessary to aid our research.
  \item To perform a thorough analysis on the collected data.
  \item To come up with a conclusion from the data analysis.
\end{itemize}


\subsection{\textbf{Scope}}
This research is aimed at programmers or computer science students at higher institutions of learning and those at work places.

\subsection{\textbf{Research Significance}}
This study is important because it aims at improving the computer science curriculum at higher institutions of learning.




\section{\textbf{Methodology}}
The proposed methodology consists of two phases, data collection and data analysis.\bigbreak
Data will be collected using ODK Collect, which will later on be uploaded to the ODK aggregate server to carry out all the required analysis. Different kinds of data (including images and GPS coordinates) will be collected, these include: 





\begin{thebibliography}{10} \bibitem{latexGuide} Do programmers work at night?, \emph{Stanford University(June 2003)}, Available at \texttt{https://leanpub.com/} \end{thebibliography}



\end{document}